\documentclass[../main/main.tex]{subfiles}

\newdate{date}{08}{10}{2020}


\begin{document}

\marginpar{ \textbf{Lecture 3.} \\  \displaydate{date}. \\ Compiled:  \today.}

\section{Basic models}

We are gonna introduce some of the basic models we will deal for the entire course.
We are assuming that we are in well-mixed populations, or homogeneous mixing. Mathematically, it is what is called mean field approximation.
In the well-mixed population assumptions we are assuming that:
\begin{itemize}
\item All individuals are equivalent, hence very one has the same probability of getting infection;
\item Every individual has the same number of contacts \( N-1 \) or on average \( \expval{k}  \).
\item Another important assumption is that we are in a closed population. Hence, the sum of the density distribution of the individuals is 1, we have no deaths or births. We are assuming that our time scale so little that we can consider population constant.
\end{itemize}

\subsection{SI model}

This simple model is the \textbf{SI} (susceptible infected). You can get the infection and once you get it you cannot recovery and you stay infected forever.
The transition is:
\begin{equation*}
  S + I \overset{\beta }{\rightarrow} I + I
\end{equation*}
The \( \beta  \) takes the speed of the spread. We can write down the equation and solve it determically:
\begin{equation*}
  \dv{s}{t} = - \beta \expval{k} si, \qquad \dv{i}{t} = \beta \expval{k} si
\end{equation*}
The basic equation is very simple. The \( i \) means the fraction of infected in the population, while \( s \) the fraction of susceptible in the population. Hence, \( si \) is the probability of contact. \( \beta  s i \) is the proabability of having one more infected.
To solve it analitically, we should remember that our population is closed hence \( s+i=1 \). See the calculus in the slides.
The result is:
\begin{equation*}
  i(t) = \frac{i_0 e^{\beta t} }{1-i_0 + i_0 e^{\beta t} }
\end{equation*}
which is a sigmoid function which always saturates at 1. We have the first part where we have the exponential growth (which is the one we have seen in the media), then at a certain point you are slowing down. The reason of slowing down is because of the term \( s i \), the probability of funding new supsceptible is going down. Finally, you reach 1 after a very long time. The value of \( \beta  \) is the one which drives the spreading. Increasing \( \beta  \) we have a faster exponential growth.
This was the simplest model.

\begin{remark}
In the course we are gonna use capital letter for integer numbers and small lecter for densities.
\end{remark}

\subsection{SIS model}

Now, let us go to the \textbf{SIS} model. This model starts to be more complicated. We have two different transitions:
\begin{equation*}
  S + I \overset{\beta }{\rightarrow } I + I, \qquad I \overset{\mu }{\rightarrow } S
\end{equation*}
whose second one is spontaneous.
The important things is that it is the simplest models in which a dynamical equilibrium can be reached. An individual could recover after the disease. There are always people infected that can propagate the disease. The \( \mu  \) is the recovery rate which determines the time-scale of the infection.
Dividing \( \beta  \) by \( \mu  \) you can rescaling all the dynamics.
The equations are exatly the same expect for a term:
\begin{equation*}
  \dv{s}{t} = - \beta \expval{k} si + \mu i, \qquad \dv{i}{t} = \beta \expval{k} si - \mu i
\end{equation*}
and you can solve them in the way of before.
Also the shape of the solution is exactly the same.
If we plot it we have the same form but with the difference that we are not getting one but \( \frac{\beta - \mu }{\beta } \). Hence, as said, we have some sort of dynamical equilibrium: the new infected are the same of the new recovery that you are getting. The population will fluctuate around this value \( \frac{\beta - \mu }{\beta } \) and enlarging \( \mu  \) will give to larging fluctuations.

It could be more instructive to study what happens at the beginning for this model. At the beginning I can assume that almost my population is composed by my susceptible \( s \sim 1 \) and the number of infected is very little \( i \ll 1 \).
Hence, we can rewrite the equation as:
\begin{equation*}
  \dv{i}{t} = \beta \expval{k} s i - \mu i \sim \beta \expval{k} i - \mu i \rightarrow i(t) \sim i_0 e^{(\beta \expval{k} - \mu  )t}
\end{equation*}
We have that if \( \beta \expval{k} < \mu   \) I have not spreading at this point, while if \( \beta \expval{k} > \mu   \) the exponential becomes positive and I have the exponential growing at the beginning.
The very important thing is that if I consider what it is happening I have two choices for the steady state:
\begin{equation*}
  \dv{i}{t} = 0 \rightarrow  \begin{cases}
   i=0 & \beta \expval{k} < \mu  \\
   i>0 & \beta \expval{k} > \mu
  \end{cases}
\end{equation*}
We have that:
\begin{equation*}
  i>0 \iff \beta > \beta_c = \frac{\mu }{ \expval{k} }
\end{equation*}
is the \textbf{epidemic threshold}. Which is telling you if the disease is gonne spread.
The epidemic threshold is the minimum value of the infection probability for which the disease survives. This is what in physics is called a second order phase transition. In this case the critical exponents are the same of the Ising model (they are in the same class of universality).
This is one of the most important quantities we are gonna studing.

What is the relation between \( R_0 \) and the epidemic threshold? We are saying that we have a critical value. Below it we have no spreading, while above we have a fraction of infected people. The two quantities are strongly correlated.
The epidemic threshold is given you the condition under which you have the spreading. Mathematically, giving a speficic model the critical model is giving the value for which \( R_0=1 \).This means that if you are above the threshold you need a minimum of infected people which is 1.  In the case of the SIS model:
\begin{equation*}
  R_0 = \frac{\beta \expval{k} }{\mu } = 1
\end{equation*}


\subsection{SIR model}
The idea is the same of the SIS, but we are adding a new states which accounts for long lasting immunity. Hence, once you got the disease you can have a long immunity. The density of the population still to be 1.
\begin{equation*}
  S + I \overset{\beta }{\rightarrow } I + I, \qquad I \overset{\mu }{\rightarrow } R
\end{equation*}
The equation are exactly the same of before but
\begin{equation*}
  \dv{s}{t} = - \beta \expval{k} si, \qquad \dv{i}{t} = \beta \expval{k} si - \mu i, \qquad  \dv{r}{t} = \mu i
\end{equation*}
It is a good point to introduce the differente regimes that you have during a spreading. Initially, at the beginning of each spreading, you have the \textbf{noisy phases} where the numbers are too small to cause a spreading, hence you have a sort of stocastic fluctuations. In most of the cases, you can end up without infected. If you stop a guy in this noisy phase, you are able to stop the disease (if it is hetherogeneous). Then, we have the \textbf{exponential growth}. Then, the disease is slowing down. Finally, you reach the steady state for the SIS, while for the SIR it disappear.

To calculate the epidemic threshold in the case of the SIR the calculations are the same of before. The result is:
\begin{equation*}
  \beta > \beta _c = \frac{\mu }{\expval{k} }
\end{equation*}

Since we can get an expression for S and I, we want to study what is the behavior at the end for \( t = \infty  \). We get that:
\begin{equation*}
  \dv{s}{r} = \frac{- \beta  \expval{k}  s}{\mu }
\end{equation*}
and you can see that the infected are disappeared. If we assume \( R_0 = 0  \) we obtain:
\begin{equation*}
  s(t) = s_0 e^{-r(t) \frac{\beta \expval{k} }{\mu }}
\end{equation*}
We cannot solve this equation directly, but we can study the behavior in the long term. At \( t=\infty  \), we have that \( i (\infty ) = 0 \) and thus \( s(\infty ) = 1 - r(\infty ) \):
\begin{equation*}
  1 - r(\infty ) - s_0 e^{-r(\infty ) \frac{\beta \expval{k} }{\mu }} = 0
\end{equation*}
and let us note that if \( s_0 \ll 1 \) the disease cannot spread.


\section{Extensions of the SIR model}
We want to modify the SIR to include something that we want in our model. We were assuming that the population was totally closed and so it always sum up to 1. This is one thing that we want to remove because it is unrealistic. We will assume that there could be births and deaths.

If we consider the demography, we see that every year there are new child that are infected by disease as Measles and Chickenpox. We expect that usually they die out over weeks.

We are assuming that \( \alpha  \) is the death rate in all the classes (deaths are not due to the disease).

We are assuming also constant population:
\begin{equation*}
  \dv{s}{t} + \dv{i}{t} + \dv{r}{t} = 0
\end{equation*}

We change our equation as:
\begin{equation*}
  \dv{s}{t} = \alpha - \beta s i - \alpha s, \qquad \dv{i}{t} = \beta s i - \mu i - \alpha i, \qquad \dv{r}{t} = \mu i - \alpha r
\end{equation*}
This complicates the study of the dynamic and for solving it you are assuming that the population is constant, hence at the equilibrium state:
\begin{equation*}
  \dv{s}{t} = \dv{i}{t} = \dv{r}{t} = 0
\end{equation*}
We obtain the equation:
\begin{equation*}
  i^* \qty[\beta s^* - (\mu + \alpha )] = 0
\end{equation*}
which is not differential anymore. The two solutions are \( i^* = 0 \) (\textbf{disease free state}) or \( s^* = \frac{\alpha + \mu }{\beta } = \frac{1}{R_0} \) which is the \textbf{endemic state}.

We can also calculate of the three values of the fraction of infected in the endemic state:
\begin{equation*}
  (s^*, i^*, r^*) = \qty( \frac{1}{R_0}, \frac{\alpha }{\beta } (R_0 -1 ), 1- \frac{1}{R_0} - \frac{\alpha }{\beta } (R_0-1))
\end{equation*}
This exists only if \( R_0>1 \).


\end{document}
