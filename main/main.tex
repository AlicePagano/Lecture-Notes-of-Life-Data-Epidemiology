\documentclass[11pt, a4paper, twoside, openright]{book}
\usepackage{subfiles}

% Algorithms
	%\usepackage{algpseudocode}
	%\usepackage{algorithm}

% Babel
\usepackage[english]{babel}

% Code writing
	%\usepackage[procnames]{listings}

% Font
\usepackage[utf8]{inputenc}
\usepackage[T1]{fontenc}
\usepackage{amssymb,amsmath,amsthm,amsfonts}
\usepackage{eucal}
\usepackage{textcomp}

% Footnote


% Hyperref
\usepackage[hyphens]{url}
\usepackage{cite}
\usepackage{hyperref}
\usepackage{nameref}

% Images
\usepackage[pdftex]{graphicx}
	%\usepackage{subfigure}
\usepackage{subfig}
\usepackage{eso-pic}
\usepackage{caption}
\usepackage{wrapfig}

% List
\usepackage{enumerate}

% SI units
\usepackage{siunitx}

% Standalone
\usepackage[subpreambles=true]{standalone}
\usepackage{import}

% Tables
\usepackage{tabularx}
\usepackage{booktabs}
\usepackage{multirow}

% TiKz and graphs
\usepackage{pgf,tikz,pgfplots}
% \usepackage{gnuplottex}
\usepackage{bm}
\usepackage{relsize}
%\usepackage[compat=1.1.0]{tikz-feynman}
\usepackage{circuitikz}

% Typeset
%\usepackage[top=2cm,bottom=2cm,left=2cm,right=2cm]{geometry}
\usepackage[top=2cm,bottom=2cm,left=2cm,right=2cm]{geometry}
\usepackage{fancyhdr}
\usepackage{indentfirst}
\usepackage{titlesec}
\usepackage{setspace}
\usepackage{xspace}
% \usepackage{parskip}  % Elimina il separatore a inizio paragrafo
\usepackage{afterpage}
\usepackage{comment}

%Python
\usepackage{xcolor}
\usepackage{listings}
\usepackage{framed}

%Per scrivere matrice identità
\usepackage{bbold}
%Per semplificazione formule
\usepackage{cancel}

%Evidenziare formule
\usepackage{empheq}
	%oppure
	%\usepackage{xcolor}
\usepackage{soul}

%Evidenziare testo con mdframed
\usepackage{mdframed}

%Note a margine
\usepackage{marginnote}

%Display data
\usepackage{datetime}

%Physics
\usepackage{physics}
%Geometry
\newgeometry{inner=20mm,
            outer=49mm,% = marginparsep + marginparwidth
                       %   + 5mm (between marginpar and page border)
            top=20mm,
            bottom=25mm,
            marginparsep=6mm,
            marginparwidth=30mm}
\makeatletter
\renewcommand{\@marginparreset}{%
  \reset@font\small
  \raggedright
  \slshape
  \@setminipage
}
\makeatother

%Atom Latex
\pgfplotsset{compat=1.15}

%%
\captionsetup[table]{font=small,labelfont={bf},skip=10pt}
\captionsetup[figure]{font=small,labelfont={bf},skip=10pt}

%intestazione pagina
\pagestyle{fancy}
\fancyhf{}
\fancyhead[RE]{\ifnum\value{chapter}>0\nouppercase{\leftmark}\fi}
\fancyhead[LE]{\small\textbf{\thepage}}
\fancyhead[LO]{\nouppercase{\rightmark}}
\fancyhead[RO]{\small\textbf{\thepage}}

%link ipertestuale per indice
\hypersetup{
	colorlinks=false,
	linkcolor=black,
	filecolor=blue,
	citecolor = blue,
	urlcolor=blue,
	}

%%%%%indent%%%
\setlength{\parindent}{15pt}
\setlength{\parskip}{0pt}


%boh
\renewcommand{\chaptermark}[1]{%
 \markboth{\MakeUppercase{%
 \chaptername}\ \thechapter.%
 \ #1}{}}


 %Python in latex
 \definecolor{codegreen}{rgb}{0,0.6,0}
\definecolor{codegray}{rgb}{0.5,0.5,0.5}
\definecolor{codepurple}{rgb}{0.58,0,0.82}
\definecolor{backcolour}{rgb}{0.95,0.95,0.92}
\definecolor{commentcolour}{rgb}{0.43,0.63,0.65}

\definecolor{shadecolor}{rgb}{0.93, 0.93, 0.93}
\definecolor{darkgreen}{rgb}{0.0, 0.5, 0.0}
\definecolor{darkred}{rgb}{0.8, 0.0, 0.0}
\definecolor{violet}{rgb}{0.55, 0.0, 0.55}

\lstdefinestyle{mystyle}{ %Stile python code
    backgroundcolor=\color{shadecolor},
    commentstyle=\color{commentcolour},
    keywordstyle=\color{darkgreen},
    numberstyle=\tiny\color{codegray},
    stringstyle=\color{darkred},
    basicstyle=\ttfamily\footnotesize,
    breakatwhitespace=false,
    breaklines=true,
    captionpos=b,
    keepspaces=true,
    numbers=left,
    numbersep=5pt,
    showspaces=false,
    showstringspaces=false,
    showtabs=false,
    tabsize=2
}

\lstset{style=mystyle}

% Derivatives
\renewcommand{\d}[0]{\mathrm{d}}
\newcommand{\dev}[2]{\displaystyle \frac{\d #1}{\d #2}}
\newcommand{\pdev}[2]{\displaystyle \frac{\partial #1}{\partial #2}}
\newcommand{\ndev}[3]{\displaystyle \frac{\d^{#3} #1}{\d #2^{#3} } }
\newcommand{\npdev}[3]{\displaystyle \frac{\partial^{#3} #1}{\partial #2^{#3} } }


%% Norms
\newcommand{\absvec}[1]{| \vec{#1} |}
\newcommand{\normvec}[1]{|\!| \vec{#1} |\!|}

\newcommand{\vmed}[1]{\left \langle #1 \right \rangle}
\newcommand{\vmedvec}[1]{\langle #1 \rangle}
\newcommand{\R}[0]{\mathbb{R}}
\renewcommand{\H}[0]{\operatorname{H}}

%Evidenziare formule
\newcommand{\mathcolorbox}[2]{\colorbox{#1}{$\displaystyle #2$}}
\newcommand{\hlfancy}[2]{\sethlcolor{#1}\hl{#2}}

%Theorem
\newtheorem{theorem}{Theorem}[section]
\newtheorem{corollary}{Corollary}[theorem]
\newtheorem{lemma}[theorem]{Lemma}

\theoremstyle{definition}
\newtheorem{definition}{Definition}%[section]
\newtheorem{exercise}{Exercise}
\newtheorem{example}{Example}

\theoremstyle{remark}
\newtheorem*{remark}{Remark}
\newtheorem{observation}{Observation}
%Evidenziare testo

\newcommand\mybox[1]{%
  \fbox{\begin{minipage}{0.9\textwidth}#1\end{minipage}}}

  %Spiegazioni/verifiche
\newenvironment{greenbox}{\begin{mdframed}[hidealllines=true,backgroundcolor=green!20,innerleftmargin=3pt,innerrightmargin=3pt]}{\end{mdframed}}

\newenvironment{bluebox}{\begin{mdframed}[hidealllines=true,backgroundcolor=blue!10,innerleftmargin=3pt,innerrightmargin=3pt]}{\end{mdframed}}

\newenvironment{yellowbox}{\begin{mdframed}[hidealllines=true,backgroundcolor=yellow!20,innerleftmargin=3pt,innerrightmargin=3pt]}{\end{mdframed}}

\newenvironment{redbox}{\begin{mdframed}[hidealllines=true,backgroundcolor=red!20,innerleftmargin=3pt,innerrightmargin=3pt]}{\end{mdframed}}

\newenvironment{orangebox}{\begin{mdframed}[hidealllines=true,backgroundcolor=orange!20,innerleftmargin=3pt,innerrightmargin=3pt]}{\end{mdframed}}

%emph equation
\newcommand*\myyellowbox[1]{%
  \colorbox{yellow!40}{\hspace{1em}#1\hspace{1em}}}
  
  
 % “Inspirational” quote at start of chapter
  \makeatletter
\renewcommand{\@chapapp}{}% Not necessary...
\newenvironment{chapquote}[2][2em]
  {\setlength{\@tempdima}{#1}%
   \def\chapquote@author{#2}%
   \parshape 1 \@tempdima \dimexpr\textwidth-2\@tempdima\relax%
   \itshape}
  {\par\normalfont\hfill--\ \chapquote@author\hspace*{\@tempdima}\par\bigskip}
\makeatother





%\usepackage{eso-pic}
%
%\newcommand\BackgroundPic{%
%\put(0,0){%
%\parbox[b][\paperheight]{\paperwidth}{%
%\vfill
%\centering
%\includegraphics[scale=1.5]{../frontespizio/back.jpg}%
%\vfill
%}}}

%\includegraphics[width=1.4\paperwidth,height=\paperheight,%
%keepaspectratio]{../frontespizio/back.jpg}%


\begin{document}

%%%%%%FRONTESPIZIO%%%%%%
%Reset the geometry of frontespizio
\newgeometry{inner=20mm,
            outer=20mm,% = marginparsep + marginparwidth
                       %   + 5mm (between marginpar and page border)
            top=20mm,
            bottom=20mm,
            marginparsep=6mm,
            marginparwidth=30mm}
\makeatletter
\renewcommand{\@marginparreset}{%
  \reset@font\small
  \raggedright
  \slshape
  \@setminipage
}
\makeatother


\frontmatter

%\pagecolor{blue} %Page color blue!70
%%%%%%FRONTESPIZIO%%%%%%
\begin{titlepage} % Suppresses headers and footers on the title page

	\centering % Centre everything on the title page
	
	\scshape % Use small caps for all text on the title page
	
	\vspace*{\baselineskip} % White space at the top of the page
	
	%------------------------------------------------
	%	Title
	%------------------------------------------------
	
	\rule{\textwidth}{1.6pt}\vspace*{-\baselineskip}\vspace*{2pt} % Thick horizontal rule
	\rule{\textwidth}{0.4pt} % Thin horizontal rule
	
	\vspace{0.75\baselineskip} % Whitespace above the title
	
	{\LARGE LECTURE NOTES\\ OF\\ LIFE DATA EPIDEMIOLOGY \\} % Title
	
	\vspace{0.75\baselineskip} % Whitespace below the title
	
	\rule{\textwidth}{0.4pt}\vspace*{-\baselineskip}\vspace{3.2pt} % Thin horizontal rule
	\rule{\textwidth}{1.6pt} % Thick horizontal rule
	
	\vspace{2\baselineskip} % Whitespace after the title block
	
	%------------------------------------------------
	%	Subtitle
	%------------------------------------------------
	
	Collection of the lectures notes of professors Chiara Polletto and Sandro Meloni. % Subtitle or further description
	
	\vspace*{3\baselineskip} % Whitespace under the subtitle
	
	%------------------------------------------------
	%	Editor(s)
	%------------------------------------------------
	
	Edited By
	
	\vspace{0.5\baselineskip} % Whitespace before the editors
	
	{\scshape\Large Alice Pagano \\} % Editor list
	
	\vspace{0.5\baselineskip} % Whitespace below the editor list
	
	\textit{The University of Padua } % Editor affiliation
	
	\vfill % Whitespace between editor names and publisher logo
	
%	%------------------------------------------------
%	%	Publisher
%	%------------------------------------------------
%	
%	\plogo % Publisher logo
%	
%	\vspace{0.3\baselineskip} % Whitespace under the publisher logo
%	
%	2017 % Publication year
%	
%	{\large publisher} % Publisher

\end{titlepage}

\clearpage{\pagestyle{empty}\cleardoublepage}


%\pagecolor{white}

%%%ABSTRACT%%%%%%%%%

%\chapter*{\centering {\normalsize Abstract}}
%\noindent In this document I have tried to reorder the notes of the “Introduction to Many-Body Theory” course held by Professor Pier Luigi Silvestrelli at the Department
%of Physics of the University of Padua during the second semester of the 2019-20 academic year of the master's degree.
%%The notes are \textbf{fully} integrated with the material provided by the professor in the Moodle platform. 
%%In addition, I will integrate them, as best as possible, with the books recommended by the professor.
%There may be formatting errors, wrong marks, missing exponents etc. If you find errors, let me know (alice.pagano@studenti.unipd.it) and I will correct them, so that this document can be a good study support.
%
%\vspace{1cm}
%\noindent Padova, \today  \hspace{6cm} Alice Pagano

\tableofcontents


\pagestyle{plain}



\newgeometry{inner=20mm,
            outer=49mm,% = marginparsep + marginparwidth
                       %   + 5mm (between marginpar and page border)
            top=20mm,
            bottom=25mm,
            marginparsep=6mm,
            marginparwidth=30mm}
\makeatletter
\renewcommand{\@marginparreset}{%
  \reset@font\small
  \raggedright
  \slshape
  \@setminipage
}
\makeatother










%
%\chapter*{Introduction}
%
%\section*{Useful informations}
%
%Reference textbook:
%\begin{itemize}
%\item Fetter A.L., Walecka J.D., "Quantum Theory of Many-Particle Systems" (1971,2003)
%\item  H. Bruns, K. Flensberg, "Many-Body Quantum theory in Condensed Matter Physics" (2002)
%\end{itemize}
%
%There are two options for the exam:
%\begin{itemize}
%\item Homeworks (detailed calculations, send the solution to the professor by a PDF file in 10 days) + "simplified" oral exam (quantitative concepts, no long calculations).
%\item Standard oral exam.
%\end{itemize}
%
%\clearpage
%\LARGE{ \textbf{Outline of the course} }
%
%\normalsize
%\begin{enumerate}
%\item ciao
%\end{enumerate}






\mainmatter
\pagestyle{fancy}

\part{Meloni's Lectures}

\subfile{../lessons/02_02-10-2020.tex}
\subfile{../lessons/03_08-10-2020.tex}
\subfile{../lessons/05_15-10-2020.tex} 
\subfile{../lessons/06_16-10-2020.tex} 
\subfile{../lessons/07_22-10-2020.tex} 
\subfile{../lessons/09_29-10-2020.tex} 
\subfile{../lessons/11_05-11-2020.tex} %To be finished

\part{Poletto's Lectures}

\backmatter
\pagestyle{plain}

%\chapter{Conclusions}

%%%BIBLIOGRAFIA%%%

\cleardoublepage
\addcontentsline{toc}{chapter}{\bibname}
\begin{thebibliography}{99}

% \bibitem{Nielsen}
% Michael A. Nielsen and Isaac L. Chuang.
% \textit{Quantum computation and quantum information}.
% United Kingdom, Cambridge University Press, 2016.
%
%\bibitem{Divincenzo}
%David P. DiVincenzo
%\textit{The Physical Implementation of Quantum Computation}.
%IBM T.J. Watson Research Center, Yorktown Heights, NY 10598 USA
%February 1, 2008.
%arXiv:quant-ph/0002077
%
%\bibitem{DocumentationQiskit}
%The Qiskit Developers.
%\textit{Qiskit API documentation}.
%Release 0.8.0, 9 March 2019.
%\url{https://qiskit.org/documentation/index.html}

\end{thebibliography}


\end{document}
