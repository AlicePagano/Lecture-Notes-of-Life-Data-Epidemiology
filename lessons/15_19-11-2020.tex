\documentclass[../main/main.tex]{subfiles}

\newdate{date}{19}{11}{2020}


\begin{document}

\marginpar{ \textbf{Lecture 15.} \\  \displaydate{date}. \\ Compiled:  \today.}

Ci sono delle aree in cui la gente tende a essere più suscettibile. Aspetto di spreading della malattia. Come le opinioni si formano e la gente pensa di diventare no vax. C'è poi l'aspetto di unire le due cose. Ci sarebbeanche l'aspetto: se la malattia aumenta la gente si ricorda quanto cattivo sia il morbillo e inizia a rivaccinarsi.

Va bene progetto diffusione di opinioni. Come

1 direzione. Modello teorico
2 direzione. Cercare dati di cluster per capire se ci sono zone maggiormente a rischio.
3 direzione. dati facebook

IDEA
Come si diffonde l'opinione no vax. 




\end{document}
